\documentclass[8pt,a4paper]{article} % Compila con lualatex o xelatex

% --- Layout & links ---
\usepackage[margin=1.5in]{geometry}
\usepackage[hidelinks]{hyperref}

% --- Fuentes (texto y matemáticas) ---
\usepackage[cmintegrals,cmbraces]{newtxmath}
\usepackage{ebgaramond-maths}
\usepackage{fontspec}
\setmainfont{EB Garamond}[
    UprightFont = * Regular,
    ItalicFont = * Italic,
    BoldFont = * SemiBold,
    BoldItalicFont = * SemiBold Italic,
]


% --- Secciones y subsecciones ---
\usepackage{titlesec}

\newfontface\bold{EB Garamond Bold}
\newfontface\bolditalic{EB Garamond  Bold Italic}
\newfontface\extrabold{EB Garamond ExtraBold}

\titleformat{\section}
  {\bold\Large}{\thesection}{0.5em}{}
\titleformat{\subsection}
  {\bold\large}{\thesubsection}{0.5em}{}


% --- Encabezado simple ---
% \usepackage{fancyhdr}
% \pagestyle{fancy}
% \fancyhf{}
% \fancyhead[L]{\small\textsc{AI in Financial Services}}
% \fancyhead[R]{\small\thepage}
% \renewcommand{\headrulewidth}{0.05pt}


% --- Espaciado ---
\usepackage{setspace}
\setstretch{1.2}
\setlength{\parskip}{0.2em}



\begin{document}
\thispagestyle{empty}
\section*{Objective}

We model the probability that an adverse reaction belongs to a given
MedDRA System Organ Class (SOC) using patient, product, dose and
administration information from the Canada Vigilance dataset. Let
\(\mathbf{x}\) be the vector of predictors and \(Y\) the SOC label:

\[
\hat{y}
  = \arg\max_{\text{soc}} \Pr\bigl(Y = \text{soc} \mid \mathbf{x}\bigr),
\qquad
Y \equiv \text{SOC\_NAME\_ENG}.
\]

\section*{Predictors (Inputs)}

\noindent
\textbf{Demographics and anthropometrics}
\begin{itemize}
  \item \texttt{GENDER\_CODE}: Encoded sex of the patient.
  \item \texttt{age\_y}: Age in years.
  \item \texttt{weight\_kg}: Body weight in kilograms.
  \item \texttt{height\_cm}: Height in centimetres.
\end{itemize}

\noindent
\textbf{Exposure and dosing}
\begin{itemize}
  \item \texttt{routeadmin\_eng}: Route of administration (e.g.\ oral, IV).
  \item \texttt{unit\_dose\_qty}: Numeric dose per administration.
  \item \texttt{dose\_unit\_eng}: Dose unit in English (e.g.\ mg, mg/kg, mL).
  \item \texttt{hours\_between\_medicament}: Time in hours between administrations.
\end{itemize}

\noindent
\textbf{Product and indication}
\begin{itemize}
  \item \texttt{indication\_name\_eng}: Therapeutic indication in English.
  \item \texttt{active\_ingredient\_name}: Active ingredient name(s) of the product.
\end{itemize}

\section*{Target (Output)}

\begin{itemize}
  \item \texttt{SOC\_NAME\_ENG}: MedDRA System Organ Class in English
        (e.g.\ ``Cardiac disorders'', ``Skin and subcutaneous tissue disorders'').
\end{itemize}

\section*{Modeling Note}

This is a supervised multi-class classification problem with potential
class imbalance across SOCs. Evaluation should therefore include
macro-averaged metrics (e.g.\ macro-F1, macro-recall) and confusion
matrices by SOC.

\end{document}